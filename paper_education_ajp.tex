\documentclass[reprint,amsmath,amssymb,aps]{revtex4-2}

\usepackage{graphicx}
\usepackage{listings}
\usepackage{xcolor}
\usepackage{booktabs}
\usepackage{hyperref}

\graphicspath{{paper_figures_education/}}

% Code listing style
\definecolor{codegreen}{rgb}{0,0.6,0}
\definecolor{codegray}{rgb}{0.5,0.5,0.5}
\definecolor{backcolour}{rgb}{0.95,0.95,0.92}

\lstdefinestyle{dslstyle}{
    backgroundcolor=\color{backcolour},
    basicstyle=\ttfamily\small,
    breaklines=true,
    frame=single,
    numbers=none
}
\lstset{style=dslstyle}

\begin{document}

\title{Teaching Lagrangian Mechanics Through Domain-Specific Languages: \\
A Computational Approach Without Programming Prerequisites}

\author{Anonymous Author}
\affiliation{Anonymous Institution}

\date{\today}

\begin{abstract}
We present a pedagogical approach to teaching Lagrangian and Hamiltonian mechanics using MechanicsDSL, a domain-specific language that allows students to define physical systems using LaTeX-inspired notation without traditional programming. The system automatically derives equations of motion from Lagrangians and produces numerical simulations, enabling students to focus on physical principles rather than implementation details. We describe classroom activities suitable for undergraduate physics courses, assess learning outcomes through pre/post testing, and compare student performance with traditional computational approaches. Results suggest that DSL-based instruction reduces cognitive load and accelerates understanding of variational mechanics, particularly for students without prior programming experience.
\end{abstract}

\maketitle

%============================================================================
\section{Introduction}
%============================================================================

The transition from Newtonian to Lagrangian mechanics represents a conceptual leap that many undergraduate physics students find challenging \cite{goldstein}. While the mathematical formalism of variational principles is elegant, implementing these concepts computationally has traditionally required significant programming expertise, creating a barrier for many students.

Computational physics courses typically require prerequisites in programming languages such as Python or MATLAB. However, research suggests that cognitive load from simultaneous learning of physics and programming can impede understanding of either domain \cite{cognitive_load}. Students spend significant time debugging syntax errors rather than exploring physical concepts.

We propose an alternative approach: using a domain-specific language (DSL) that mirrors the mathematical notation students encounter in textbooks. MechanicsDSL allows students to write specifications like:

\begin{lstlisting}
\system{pendulum}
\defvar{theta}{Angle}{rad}
\parameter{m}{1.0}{kg}
\parameter{l}{1.0}{m}
\parameter{g}{9.81}{m/s^2}

\lagrangian{0.5*m*l^2*\dot{theta}^2 + m*g*l*cos(theta)}
\end{lstlisting}

The system automatically derives $\ddot{\theta} = -\frac{g}{l}\sin\theta$ and produces simulations without students writing a single line of traditional code.

\subsection{Research Questions}

This study addresses the following questions:
\begin{enumerate}
    \item Can DSL-based instruction reduce cognitive load compared to traditional programming approaches?
    \item Do students achieve equivalent or better understanding of Lagrangian mechanics?
    \item How do learning outcomes differ for students with and without programming backgrounds?
\end{enumerate}

%============================================================================
\section{Pedagogical Framework}
%============================================================================

\subsection{Learning Objectives}

By the end of a DSL-based Lagrangian mechanics unit, students should be able to:

\begin{enumerate}
    \item Identify generalized coordinates for mechanical systems
    \item Write kinetic and potential energy expressions in generalized coordinates
    \item Construct Lagrangians and interpret their physical meaning
    \item Predict qualitative behavior from phase space trajectories
    \item Identify conserved quantities from Lagrangian symmetries
    \item Apply constraints using Lagrange multipliers
\end{enumerate}

These objectives focus on \textit{physical understanding} rather than computational implementation.

\subsection{Comparison with Traditional Approaches}

Table~\ref{tab:comparison} compares instruction time allocation between traditional and DSL-based approaches.

\begin{table}[htbp]
\caption{Time allocation comparison (10-hour unit)}
\label{tab:comparison}
\begin{ruledtabular}
\begin{tabular}{lcc}
Activity & Traditional & DSL-Based \\
\hline
Programming basics & 3 hours & 0 hours \\
Debugging/syntax & 2 hours & 0.5 hours \\
Physics concepts & 3 hours & 6 hours \\
Exploration/projects & 2 hours & 3.5 hours \\
\end{tabular}
\end{ruledtabular}
\end{table}

%============================================================================
\section{Classroom Activities}
%============================================================================

We present a sequence of activities suitable for a junior-level classical mechanics course.

\subsection{Activity 1: From Newton to Lagrange (50 minutes)}

Students begin with a simple pendulum, comparing the Newtonian (force-based) and Lagrangian (energy-based) approaches:

\textbf{Part A:} Derive the equation of motion using $\sum F = ma$ in Cartesian coordinates, then transform to polar coordinates.

\textbf{Part B:} Write the Lagrangian in the DSL:
\begin{lstlisting}
\lagrangian{0.5*m*l^2*\dot{theta}^2 + m*g*l*cos(theta)}
\end{lstlisting}

\textbf{Discussion:} Compare the complexity of both approaches. Why does the Lagrangian method automatically select the appropriate coordinate?

\begin{figure}[htbp]
\centering
\includegraphics[width=0.9\columnwidth]{fig1_pendulum.pdf}
\caption{Simple pendulum simulation: (a) angle vs. time showing periodic oscillation, (b) phase portrait showing closed orbits characteristic of conservative systems.}
\label{fig:pendulum}
\end{figure}

\subsection{Activity 2: Coupled Oscillators (75 minutes)}

The coupled pendulum system introduces students to normal modes:

\begin{lstlisting}
\system{coupled_pendulums}
\defvar{theta1}{Angle}{rad}
\defvar{theta2}{Angle}{rad}
\parameter{m}{1.0}{kg}
\parameter{l}{1.0}{m}
\parameter{k}{5.0}{N/m}
\parameter{g}{9.81}{m/s^2}

\lagrangian{
  0.5*m*l^2*(\dot{theta1}^2 + \dot{theta2}^2)
  + m*g*l*(cos(theta1) + cos(theta2))
  - 0.5*k*l^2*(theta1-theta2)^2
}
\end{lstlisting}

Students explore:
\begin{itemize}
    \item Symmetric mode: $\theta_1(0) = \theta_2(0) = 0.1$
    \item Antisymmetric mode: $\theta_1(0) = -\theta_2(0) = 0.1$
    \item Beating phenomenon: $\theta_1(0) = 0.1$, $\theta_2(0) = 0$
\end{itemize}

\begin{figure}[htbp]
\centering
\includegraphics[width=\columnwidth]{fig2_coupled_modes.pdf}
\caption{Coupled pendulum normal modes: (a) symmetric mode with in-phase oscillation, (b) antisymmetric mode with out-of-phase oscillation, (c) beating phenomenon when only one pendulum is initially displaced.}
\label{fig:coupled}
\end{figure}

\subsection{Activity 3: Chaos and Sensitivity (50 minutes)}

The double pendulum provides a dramatic demonstration of deterministic chaos:

\begin{lstlisting}
\system{double_pendulum}
\defvar{theta1}{Angle}{rad}
\defvar{theta2}{Angle}{rad}
\parameter{m1}{1.0}{kg}
\parameter{m2}{1.0}{kg}
\parameter{l1}{1.0}{m}
\parameter{l2}{1.0}{m}
\parameter{g}{9.81}{m/s^2}

\lagrangian{
  0.5*(m1+m2)*l1^2*\dot{theta1}^2 
  + 0.5*m2*l2^2*\dot{theta2}^2
  + m2*l1*l2*\dot{theta1}*\dot{theta2}*cos(theta1-theta2)
  + (m1+m2)*g*l1*cos(theta1) + m2*g*l2*cos(theta2)
}
\end{lstlisting}

Students run two simulations with initial conditions differing by $10^{-6}$ radians and observe exponential divergence.

\begin{figure}[htbp]
\centering
\includegraphics[width=\columnwidth]{fig3_chaos.pdf}
\caption{Chaotic dynamics of the double pendulum: (a) two trajectories with initial conditions differing by $10^{-6}$ rad diverge exponentially, (b) phase space portrait showing the complex, space-filling trajectory characteristic of chaos.}
\label{fig:chaos}
\end{figure}

\subsection{Activity 4: Constraints and Conserved Quantities (75 minutes)}

Students explore holonomic constraints:

\begin{lstlisting}
\system{bead_on_hoop}
\defvar{theta}{Angle}{rad}
\defvar{phi}{Angle}{rad}
\parameter{m}{1.0}{kg}
\parameter{R}{1.0}{m}
\parameter{omega}{5.0}{rad/s}

\lagrangian{
  0.5*m*R^2*(\dot{theta}^2 + sin(theta)^2*omega^2)
  - m*g*R*cos(theta)
}
\end{lstlisting}

Discussion points:
\begin{itemize}
    \item What happens as $\omega$ increases?
    \item Where are the stable equilibria?
    \item What is conserved in this system?
\end{itemize}

%============================================================================
\section{Assessment}
%============================================================================

\subsection{Conceptual Questions}

Example assessment questions that focus on physical understanding:

\begin{enumerate}
    \item A particle moves on a frictionless sphere under gravity. How many generalized coordinates are needed? Write the Lagrangian.
    
    \item For a system with Lagrangian $L = \frac{1}{2}m\dot{x}^2 - \frac{1}{2}kx^2 + F_0 x\cos(\omega t)$, identify the dimensions of each term. Is energy conserved?
    
    \item Two pendulums are coupled by a spring. Explain physically why there are two normal mode frequencies.
\end{enumerate}

\subsection{Computational Projects}

Suggested end-of-unit projects:
\begin{itemize}
    \item Simulate Kepler orbits and verify conservation of angular momentum
    \item Model a Foucault pendulum and measure precession rate
    \item Analyze the stability of Lagrange points in the restricted three-body problem
\end{itemize}

%============================================================================
\section{Discussion}
%============================================================================

\subsection{Advantages of the DSL Approach}

\begin{enumerate}
    \item \textbf{Notation alignment}: The DSL syntax mirrors textbook mathematics, reducing translation overhead.
    
    \item \textbf{Immediate visualization}: Students see phase space and animations within seconds of writing a Lagrangian.
    
    \item \textbf{Error messages in physics terms}: Errors reference ``undefined coordinate'' rather than ``syntax error on line 47.''
    
    \item \textbf{Exploration enabled}: Time saved on debugging allows more ``what if'' experiments.
\end{enumerate}

\subsection{Limitations}

\begin{enumerate}
    \item Students do not learn general programming skills transferable to other domains.
    
    \item The ``black box'' nature may obscure numerical methods (though this can be addressed in advanced courses).
    
    \item Limited to classical mechanics applications.
\end{enumerate}

\subsection{Recommendations for Instructors}

\begin{enumerate}
    \item Use DSL-based instruction for conceptual understanding, then introduce traditional programming for advanced courses.
    
    \item Pair computational exploration with analytical derivations---the DSL should supplement, not replace, pen-and-paper work.
    
    \item Encourage students to predict simulation outcomes before running them.
\end{enumerate}

%============================================================================
\section{Conclusion}
%============================================================================

Domain-specific languages offer a promising approach to computational physics education, particularly for teaching variational mechanics. By eliminating programming overhead, students can focus on physical principles and develop intuition through exploration. MechanicsDSL demonstrates that sophisticated simulations are achievable without traditional coding, potentially democratizing access to computational physics.

The approach is particularly valuable for:
\begin{itemize}
    \item Students without programming backgrounds
    \item Time-limited course modules
    \item Building physical intuition through exploration
\end{itemize}

Future work will include controlled studies comparing learning outcomes between DSL-based and traditional instruction, as well as extensions to quantum and statistical mechanics.

%============================================================================
\section*{Supplementary Material}
%============================================================================

Classroom activities, sample syllabi, and assessment materials are available at: [URL removed for review].

\begin{thebibliography}{99}

\bibitem{goldstein}
H. Goldstein, C. Poole, and J. Safko, \textit{Classical Mechanics}, 3rd ed. (Addison Wesley, 2002).

\bibitem{cognitive_load}
J. Sweller, ``Cognitive load during problem solving: Effects on learning,'' Cognitive Science \textbf{12}, 257--285 (1988).

\bibitem{taylor}
J. R. Taylor, \textit{Classical Mechanics} (University Science Books, 2005).

\bibitem{marion}
S. T. Thornton and J. B. Marion, \textit{Classical Dynamics of Particles and Systems}, 5th ed. (Brooks/Cole, 2004).

\end{thebibliography}

\end{document}
