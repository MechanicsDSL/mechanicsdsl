\documentclass[12pt]{letter}

\usepackage{geometry}
\geometry{margin=1in}
\usepackage{hyperref}

\signature{Noah Parsons\\Independent Researcher\\Newcastle, WY, USA}
\address{Noah Parsons\\Newcastle, WY, USA\\contact@mechanicsdsl.org}

\begin{document}

\begin{letter}{Editorial Office\\Computer Physics Communications\\Elsevier}

\opening{Dear Editors,}

I am pleased to submit the manuscript entitled ``\textbf{MechanicsDSL: A Compiler Framework for Symbolic-Numerical Classical Mechanics Simulation}'' for consideration for publication in \textit{Computer Physics Communications}.

\textbf{Summary.} This manuscript presents MechanicsDSL, a domain-specific language and compiler framework that automates the complete pipeline from human-readable physics notation to optimized numerical simulation. The system enables physicists and students to define mechanical systems using LaTeX-inspired syntax without programming expertise, automatically deriving equations of motion and generating efficient numerical solvers.

\textbf{Key Contributions:}
\begin{enumerate}
    \item A formal algebraic characterization of physics DSL compilation as a chain of typed operators
    \item A novel robust algorithm for symbolic acceleration extraction from Euler-Lagrange equations
    \item A comprehensive implementation covering 17 classical mechanics domains with code generation for 11 target platforms
    \item Rigorous validation against analytical solutions demonstrating errors below $10^{-8}$
\end{enumerate}

\textbf{Relevance to CPC.} This work directly addresses CPC's mission of publishing computational physics methods and software. MechanicsDSL bridges symbolic derivation (a traditionally manual process) with numerical simulation, providing a reproducible and validated framework for mechanics simulation. The software is freely available under the MIT license.

\textbf{Software Availability.}
\begin{itemize}
    \item Repository: \url{https://github.com/MechanicsDSL/mechanicsdsl}
    \item Package: \texttt{pip install mechanicsdsl-core}
    \item Documentation: \url{https://mechanicsdsl.readthedocs.io}
    \item DOI: 10.5281/zenodo.17771040
\end{itemize}

\textbf{Related Submission.} A software description paper has been submitted to the Journal of Open Source Software (JOSS). The JOSS paper focuses on software availability and usage, while this CPC manuscript provides the formal mathematical foundations, novel algorithms, and comprehensive validation that complement without duplicating the JOSS contribution.

\textbf{Suggested Reviewers.}
\begin{itemize}
    \item Prof.~Peter Fritzson (Link\"oping University) -- Expert in Modelica and equation-based modeling
    \item Dr.~Jason Moore (TU Delft) -- Developer of SymPy mechanics module
    \item Prof.~Anders Logg (Simula Research Laboratory) -- FEniCS developer, computational physics
\end{itemize}

Thank you for considering this manuscript. I look forward to your response.

\closing{Sincerely,}

\end{letter}

\end{document}
