\documentclass[preprint,12pt]{elsarticle}

\usepackage{amsmath,amssymb,amsthm}
\usepackage{mathtools}
\usepackage{tikz-cd}
\usepackage{algorithm}
\usepackage{algpseudocode}
\usepackage{listings}
\usepackage{xcolor}
\usepackage{hyperref}

% Theorem environments
\newtheorem{theorem}{Theorem}[section]
\newtheorem{definition}[theorem]{Definition}
\newtheorem{lemma}[theorem]{Lemma}
\newtheorem{corollary}[theorem]{Corollary}

% Custom operators
\DeclareMathOperator{\Sym}{Sym}
\DeclareMathOperator{\dom}{dom}
\DeclareMathOperator{\cod}{cod}
\DeclareMathOperator{\fresh}{fresh}
\newcommand{\MDSL}{\Sigma_{\text{MDSL}}}
\newcommand{\Lex}{\mathcal{L}}
\newcommand{\Parse}{\mathcal{P}}
\newcommand{\Symb}{\mathcal{S}}
\newcommand{\Deriv}{\mathcal{D}_{EL}}
\newcommand{\Accel}{\mathcal{A}}
\newcommand{\Num}{\mathcal{N}}
\newcommand{\Integ}{\mathcal{I}}
\newcommand{\Comp}{\mathcal{C}}
\newcommand{\Ham}{\mathcal{H}}
\newcommand{\Cons}{\mathcal{G}}
\newcommand{\Ray}{\mathcal{R}}

% Code listing style
\lstdefinestyle{dsl}{
    basicstyle=\ttfamily\small,
    keywordstyle=\color{blue},
    commentstyle=\color{gray},
    stringstyle=\color{red},
    breaklines=true,
    frame=single
}

\begin{document}

\begin{frontmatter}

\title{A Formal Semantics for Physics Domain-Specific Languages:\\
The MechanicsDSL Compilation Framework}

\author[inst1]{Noah Parsons}
\affiliation[inst1]{organization={Independent Researcher}}

\begin{abstract}
We present a formal mathematical framework describing MechanicsDSL, a domain-specific language for classical mechanics simulation. We introduce a compositional semantics where the compilation pipeline is expressed as a chain of well-defined operators mapping LaTeX-inspired source code to numerical phase-space trajectories. The framework formalizes the complete transformation from human-readable physics notation through symbolic derivation to numerical integration. We prove correctness properties and characterize the preservation of physical invariants through the compilation process. The formalism provides a rigorous foundation for reasoning about physics DSL implementations and their correctness.
\end{abstract}

\begin{keyword}
Domain-specific languages \sep Formal semantics \sep Computational physics \sep Lagrangian mechanics \sep Symbolic computation
\end{keyword}

\end{frontmatter}

%============================================================================
\section{Introduction}
%============================================================================

Domain-specific languages (DSLs) for scientific computing enable physicists and engineers to express problems in notation close to their domain expertise. However, formal semantic foundations for such languages remain underdeveloped compared to general-purpose programming languages.

MechanicsDSL addresses this gap for classical mechanics, providing:
\begin{itemize}
    \item A LaTeX-inspired syntax for defining physical systems
    \item Automatic derivation of equations of motion
    \item Numerical simulation with adaptive integration
    \item Code generation for multiple target platforms
\end{itemize}

\textbf{Contributions.} This paper makes the following contributions:
\begin{enumerate}
    \item A formal algebraic characterization of the DSL syntax space $\MDSL$
    \item Compositional operators describing each compilation stage
    \item Correctness theorems relating DSL programs to physical behavior
    \item A novel algorithm for robust symbolic acceleration extraction
\end{enumerate}

%============================================================================
\section{The MechanicsDSL Syntax Algebra}
%============================================================================

\begin{definition}[DSL Syntactic Space]
The \textbf{MechanicsDSL syntactic algebra} $\MDSL$ is the free algebra generated by:
\begin{equation}
\boxed{\MDSL = \langle \mathcal{V}, \mathcal{P}, \mathcal{E}, \mathcal{C}, \mathcal{I} \mid R \rangle}
\end{equation}
where:
\begin{itemize}
    \item $\mathcal{V}$ = variable declarations (\texttt{\textbackslash defvar})
    \item $\mathcal{P}$ = parameter definitions (\texttt{\textbackslash parameter})
    \item $\mathcal{E}$ = energy expressions (\texttt{\textbackslash lagrangian}, \texttt{\textbackslash hamiltonian})
    \item $\mathcal{C}$ = constraint expressions (\texttt{\textbackslash constraint})
    \item $\mathcal{I}$ = initial conditions (\texttt{\textbackslash initial})
    \item $R$ = production rules defining the grammar
\end{itemize}
\end{definition}

\begin{definition}[Well-Formedness]
A program $\sigma \in \MDSL$ is \textbf{well-formed} if all free symbols in expressions are either declared variables or defined parameters:
\begin{equation}
\mathcal{F}(\sigma) \subseteq \mathcal{D}(\sigma) \cup \mathcal{P}(\sigma)
\end{equation}
\end{definition}

\begin{definition}[Coordinate Projection]
The \textbf{coordinate extraction operator} $\pi_q : \MDSL \to \mathcal{Q}^n$ extracts generalized coordinates:
\begin{equation}
\pi_q(\sigma) = \mathcal{D}(\sigma) \setminus \mathcal{K}(\sigma)
\end{equation}
where $\mathcal{K}(\sigma)$ denotes constant parameters.
\end{definition}

%============================================================================
\section{The Compilation Pipeline}
%============================================================================

The MechanicsDSL compiler is characterized as a composition of operators, each with well-defined domain and codomain.

\subsection{Lexical Analysis}

\begin{definition}[Tokenization Operator]
The \textbf{lexer} $\Lex$ maps source strings to token sequences:
\begin{equation}
\Lex : \text{String} \to \mathcal{T}^*
\end{equation}
where $\mathcal{T} = \{\texttt{CMD}, \texttt{ID}, \texttt{NUM}, \texttt{OP}, \texttt{GREEK}, \ldots\}$.
\end{definition}

\subsection{Parsing}

\begin{definition}[AST Construction]
The \textbf{parser} $\Parse$ constructs abstract syntax trees:
\begin{equation}
\Parse : \mathcal{T}^* \to \mathcal{A}_{\text{MDSL}}
\end{equation}
where $\mathcal{A}_{\text{MDSL}}$ is defined inductively:
\begin{align}
\mathcal{A} ::= &\; \text{Num}(v) \mid \text{Id}(s) \mid \text{Greek}(\alpha) \\
&\mid \text{Bin}(\mathcal{A}, \otimes, \mathcal{A}) \mid \text{Un}(\ominus, \mathcal{A}) \\
&\mid \text{Der}(\mathcal{A}, k) \mid \text{Func}(f, \mathcal{A}^*)
\end{align}
\end{definition}

\subsection{Symbolic Transformation}

\begin{definition}[Symbolic Mapping]
The \textbf{symbolic engine} $\Symb$ maps AST nodes to symbolic expressions:
\begin{equation}
\Symb : \mathcal{A}_{\text{MDSL}} \to \Sym(\mathbb{R})
\end{equation}
defined recursively:
\begin{align}
\Symb[\text{Num}(v)] &= v \in \mathbb{R} \\
\Symb[\text{Id}(s)] &= \Phi(s) \quad \text{(symbol registry)} \\
\Symb[\text{Bin}(a, \otimes, b)] &= \Symb[a] \;\widetilde{\otimes}\; \Symb[b] \\
\Symb[\text{Der}(a, k)] &= \frac{d^k}{dt^k}\Symb[a]
\end{align}
\end{definition}

\subsection{Equation Derivation}

\begin{definition}[Euler-Lagrange Operator]
Given Lagrangian $L$ and coordinates $\mathcal{Q} = \{q_1, \ldots, q_n\}$:
\begin{equation}
\Deriv : \Sym(\mathbb{R}) \times \mathcal{Q}^n \to \Sym(\mathbb{R})^n
\end{equation}
such that:
\begin{equation}
\Deriv(L, q_i) = \frac{d}{dt}\left(\frac{\partial L}{\partial \dot{q}_i}\right) - \frac{\partial L}{\partial q_i}
\end{equation}
\end{definition}

This requires a \textbf{functionalization-defunctionalization} process:
\begin{align}
\mathcal{F}_t &: q \mapsto q(t), \quad \dot{q} \mapsto \frac{dq}{dt} \\
\mathcal{F}_t^{-1} &: \frac{d^2q}{dt^2} \mapsto \ddot{q}, \quad \frac{dq}{dt} \mapsto \dot{q}
\end{align}

\subsection{Acceleration Extraction}

This section describes a \textbf{novel contribution} of MechanicsDSL.

\begin{definition}[Linear Extraction Operator]
For equations linear in second derivatives:
\begin{equation}
\boxed{\Accel : E \mapsto \ddot{q} = -\frac{E\big|_{\ddot{q}=0}}{\partial E / \partial \ddot{q}}}
\end{equation}
\end{definition}

\begin{algorithm}
\caption{Robust Acceleration Extraction}
\label{alg:accel}
\begin{algorithmic}[1]
\Require Equation $E$, coordinate $q$, acceleration symbol $\ddot{q}$
\Ensure Solved acceleration expression
\State \textbf{Step 1: Symbol Search}
\For{each atom $a \in E$}
    \If{$a$ is second-order time derivative matching $q$}
        \State $E \leftarrow E[\,a \mapsto \ddot{q}\,]$
    \EndIf
\EndFor
\State \textbf{Step 2: Linear Extraction}
\State $M \leftarrow \partial E / \partial \ddot{q}$ \Comment{Mass coefficient}
\State $B \leftarrow E\big|_{\ddot{q}=0}$ \Comment{Forcing terms}
\If{$M \neq 0$}
    \State \Return $-B/M$
\Else
    \State \Return $\text{solve}(E = 0, \ddot{q})$ \Comment{Fallback}
\EndIf
\end{algorithmic}
\end{algorithm}

\begin{theorem}[Extraction Correctness]
For physically valid Lagrangians with positive-definite kinetic energy:
\begin{equation}
\det(M) \neq 0 \implies \Accel \text{ is well-defined and unique}
\end{equation}
\end{theorem}

\subsection{Numerical Compilation}

\begin{definition}[Lambdification Operator]
\begin{equation}
\Num : \Sym(\mathbb{R}) \to (\mathbb{R}^{2n+1} \to \mathbb{R})
\end{equation}
Maps symbolic expressions to callable numerical functions via index mapping:
\begin{equation}
\iota(q_i) = 2(i-1), \quad \iota(\dot{q}_i) = 2(i-1) + 1, \quad \iota(t) = -1
\end{equation}
\end{definition}

\subsection{Integration}

\begin{definition}[Flow Operator]
The \textbf{integration operator} solves initial value problems:
\begin{equation}
\Integ : (\mathbb{R} \times \Gamma \to \Gamma) \times \Gamma \times \mathbb{R}^2 \to C(\mathbb{R}; \Gamma)
\end{equation}
producing continuous trajectories in phase space $\Gamma = \mathbb{R}^{2n}$.
\end{definition}

%============================================================================
\section{The Complete Compiler}
%============================================================================

\begin{definition}[MechanicsDSL Compiler]
The complete compilation chain:
\begin{equation}
\boxed{\Comp = \Integ \circ \Num \circ \Accel \circ \Deriv \circ \Symb \circ \Parse \circ \Lex}
\end{equation}
\end{definition}

This can be visualized as a commutative diagram:

\begin{equation}
\begin{tikzcd}[column sep=small]
\text{String} \arrow[r, "\Lex"] & 
\mathcal{T}^* \arrow[r, "\Parse"] & 
\mathcal{A} \arrow[r, "\Symb"] & 
\Sym(\mathbb{R}) \arrow[d, "\Deriv"] \\
C(\mathbb{R}; \Gamma) & 
(\mathbb{R}^{2n+1} \to \mathbb{R}) \arrow[l, "\Integ"] & 
\Sym(\mathbb{R})^n \arrow[l, "\Num"] & 
\Sym(\mathbb{R})^n \arrow[l, "\Accel"]
\end{tikzcd}
\end{equation}

\begin{theorem}[Functoriality]
$\Comp$ defines a functor:
\begin{equation}
\Comp : \textbf{MDSL} \to \textbf{Dyn}
\end{equation}
from the category of well-formed DSL programs to dynamical systems.
\end{theorem}

%============================================================================
\section{Conservation and Correctness Properties}
%============================================================================

\begin{theorem}[Energy Conservation]
For time-independent Lagrangians, the compiled system preserves total energy:
\begin{equation}
\frac{d}{dt}\mathcal{E}(\mathbf{y}(t)) = 0
\end{equation}
where $\mathcal{E} = \sum_i p_i\dot{q}_i - L$ is the Hamiltonian.
\end{theorem}

\begin{theorem}[Semantic Correctness]
For a DSL program $\sigma$ encoding physical system $\Phi$:
\begin{equation}
\|\Comp(\sigma)(t) - \Phi(t)\| \leq \epsilon(h, \text{tol})
\end{equation}
where $\epsilon$ depends on step size and tolerance parameters.
\end{theorem}

\begin{theorem}[Symplectic Preservation]
For Hamiltonian systems, using appropriate integrators:
\begin{equation}
\Integ^* \omega = \omega
\end{equation}
where $\omega = \sum_i dq_i \wedge dp_i$ is the symplectic form.
\end{theorem}

%============================================================================
\section{Extended Operators}
%============================================================================

\subsection{Hamiltonian Formulation}
\begin{equation}
\Ham : \Sym(\mathbb{R}) \times \mathcal{Q}^n \to \Sym(\mathbb{R})
\end{equation}
implements the Legendre transform:
\begin{equation}
\Ham(L, \mathcal{Q}) = \sum_{q \in \mathcal{Q}} p_q \cdot \dot{q} - L
\end{equation}

\subsection{Constraint Augmentation}
\begin{equation}
\Cons : \Sym(\mathbb{R}) \times \Sym(\mathbb{R})^m \to \Sym(\mathbb{R})
\end{equation}
\begin{equation}
\Cons(L, \{g_k\}) = L + \sum_k \lambda_k g_k
\end{equation}

\subsection{Dissipation}
\begin{equation}
\Ray : \Sym(\mathbb{R})^n \to \Sym(\mathbb{R})^n
\end{equation}
augments equations with Rayleigh dissipation terms.

%============================================================================
\section{Related Work}
%============================================================================

Several frameworks address physics DSLs:

\begin{itemize}
    \item \textbf{Modelica} \cite{modelica}: Equation-based modeling with broader scope but less physics-specific notation
    \item \textbf{FEniCS} \cite{fenics}: Finite element focus with variational formulations
    \item \textbf{SymPy mechanics} \cite{sympy}: Library approach without dedicated DSL syntax
\end{itemize}

MechanicsDSL differs by providing LaTeX-inspired notation specifically for classical mechanics with automatic Euler-Lagrange/Hamilton derivation.

%============================================================================
\section{Conclusion}
%============================================================================

We have presented a formal semantics for MechanicsDSL expressed as a composition of well-typed operators. The key contributions are:

\begin{enumerate}
    \item Algebraic characterization of physics DSL syntax
    \item Compositional operator semantics for the compilation pipeline
    \item Novel robust acceleration extraction algorithm
    \item Correctness theorems connecting syntax to physical behavior
\end{enumerate}

This framework provides a foundation for:
\begin{itemize}
    \item Proving compiler correctness
    \item Analyzing numerical error propagation
    \item Extending the DSL with new physics domains
    \item Comparing physics DSL implementations
\end{itemize}

%============================================================================
% References
%============================================================================

\begin{thebibliography}{99}

\bibitem{modelica}
Fritzson, P. (2014). \textit{Principles of Object-Oriented Modeling and Simulation with Modelica 3.3}. Wiley-IEEE Press.

\bibitem{fenics}
Logg, A., Mardal, K.-A., \& Wells, G. (2012). \textit{Automated Solution of Differential Equations by the Finite Element Method}. Springer.

\bibitem{sympy}
Meurer, A., et al. (2017). SymPy: symbolic computing in Python. \textit{PeerJ Computer Science}, 3:e103.

\bibitem{goldstein}
Goldstein, H., Poole, C., \& Safko, J. (2002). \textit{Classical Mechanics} (3rd ed.). Addison Wesley.

\bibitem{lanczos}
Lanczos, C. (1970). \textit{The Variational Principles of Mechanics}. Dover.

\end{thebibliography}

\end{document}
